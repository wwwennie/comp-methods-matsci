\documentclass[11pt]{article}

    \usepackage[breakable]{tcolorbox}
    \usepackage{parskip} % Stop auto-indenting (to mimic markdown behaviour)
    

    % Basic figure setup, for now with no caption control since it's done
    % automatically by Pandoc (which extracts ![](path) syntax from Markdown).
    \usepackage{graphicx}
    % Maintain compatibility with old templates. Remove in nbconvert 6.0
    \let\Oldincludegraphics\includegraphics
    % Ensure that by default, figures have no caption (until we provide a
    % proper Figure object with a Caption API and a way to capture that
    % in the conversion process - todo).
    \usepackage{caption}
    \DeclareCaptionFormat{nocaption}{}
    \captionsetup{format=nocaption,aboveskip=0pt,belowskip=0pt}

    \usepackage{float}
    \floatplacement{figure}{H} % forces figures to be placed at the correct location
    \usepackage{xcolor} % Allow colors to be defined
    \usepackage{enumerate} % Needed for markdown enumerations to work
    \usepackage{geometry} % Used to adjust the document margins
    \usepackage{amsmath} % Equations
    \usepackage{amssymb} % Equations
    \usepackage{textcomp} % defines textquotesingle
    % Hack from http://tex.stackexchange.com/a/47451/13684:
    \AtBeginDocument{%
        \def\PYZsq{\textquotesingle}% Upright quotes in Pygmentized code
    }
    \usepackage{upquote} % Upright quotes for verbatim code
    \usepackage{eurosym} % defines \euro

    \usepackage{iftex}
    \ifPDFTeX
        \usepackage[T1]{fontenc}
        \IfFileExists{alphabeta.sty}{
              \usepackage{alphabeta}
          }{
              \usepackage[mathletters]{ucs}
              \usepackage[utf8x]{inputenc}
          }
    \else
        \usepackage{fontspec}
        \usepackage{unicode-math}
    \fi

    \usepackage{fancyvrb} % verbatim replacement that allows latex
    \usepackage{grffile} % extends the file name processing of package graphics 
                         % to support a larger range
    \makeatletter % fix for old versions of grffile with XeLaTeX
    \@ifpackagelater{grffile}{2019/11/01}
    {
      % Do nothing on new versions
    }
    {
      \def\Gread@@xetex#1{%
        \IfFileExists{"\Gin@base".bb}%
        {\Gread@eps{\Gin@base.bb}}%
        {\Gread@@xetex@aux#1}%
      }
    }
    \makeatother
    \usepackage[Export]{adjustbox} % Used to constrain images to a maximum size
    \adjustboxset{max size={0.9\linewidth}{0.9\paperheight}}

    % The hyperref package gives us a pdf with properly built
    % internal navigation ('pdf bookmarks' for the table of contents,
    % internal cross-reference links, web links for URLs, etc.)
    \usepackage{hyperref}
    % The default LaTeX title has an obnoxious amount of whitespace. By default,
    % titling removes some of it. It also provides customization options.
    \usepackage{titling}
    \usepackage{longtable} % longtable support required by pandoc >1.10
    \usepackage{booktabs}  % table support for pandoc > 1.12.2
    \usepackage{array}     % table support for pandoc >= 2.11.3
    \usepackage{calc}      % table minipage width calculation for pandoc >= 2.11.1
    \usepackage[inline]{enumitem} % IRkernel/repr support (it uses the enumerate* environment)
    \usepackage[normalem]{ulem} % ulem is needed to support strikethroughs (\sout)
                                % normalem makes italics be italics, not underlines
    \usepackage{mathrsfs}
    

    
    % Colors for the hyperref package
    \definecolor{urlcolor}{rgb}{0,.145,.698}
    \definecolor{linkcolor}{rgb}{.71,0.21,0.01}
    \definecolor{citecolor}{rgb}{.12,.54,.11}

    % ANSI colors
    \definecolor{ansi-black}{HTML}{3E424D}
    \definecolor{ansi-black-intense}{HTML}{282C36}
    \definecolor{ansi-red}{HTML}{E75C58}
    \definecolor{ansi-red-intense}{HTML}{B22B31}
    \definecolor{ansi-green}{HTML}{00A250}
    \definecolor{ansi-green-intense}{HTML}{007427}
    \definecolor{ansi-yellow}{HTML}{DDB62B}
    \definecolor{ansi-yellow-intense}{HTML}{B27D12}
    \definecolor{ansi-blue}{HTML}{208FFB}
    \definecolor{ansi-blue-intense}{HTML}{0065CA}
    \definecolor{ansi-magenta}{HTML}{D160C4}
    \definecolor{ansi-magenta-intense}{HTML}{A03196}
    \definecolor{ansi-cyan}{HTML}{60C6C8}
    \definecolor{ansi-cyan-intense}{HTML}{258F8F}
    \definecolor{ansi-white}{HTML}{C5C1B4}
    \definecolor{ansi-white-intense}{HTML}{A1A6B2}
    \definecolor{ansi-default-inverse-fg}{HTML}{FFFFFF}
    \definecolor{ansi-default-inverse-bg}{HTML}{000000}

    % common color for the border for error outputs.
    \definecolor{outerrorbackground}{HTML}{FFDFDF}

    % commands and environments needed by pandoc snippets
    % extracted from the output of `pandoc -s`
    \providecommand{\tightlist}{%
      \setlength{\itemsep}{0pt}\setlength{\parskip}{0pt}}
    \DefineVerbatimEnvironment{Highlighting}{Verbatim}{commandchars=\\\{\}}
    % Add ',fontsize=\small' for more characters per line
    \newenvironment{Shaded}{}{}
    \newcommand{\KeywordTok}[1]{\textcolor[rgb]{0.00,0.44,0.13}{\textbf{{#1}}}}
    \newcommand{\DataTypeTok}[1]{\textcolor[rgb]{0.56,0.13,0.00}{{#1}}}
    \newcommand{\DecValTok}[1]{\textcolor[rgb]{0.25,0.63,0.44}{{#1}}}
    \newcommand{\BaseNTok}[1]{\textcolor[rgb]{0.25,0.63,0.44}{{#1}}}
    \newcommand{\FloatTok}[1]{\textcolor[rgb]{0.25,0.63,0.44}{{#1}}}
    \newcommand{\CharTok}[1]{\textcolor[rgb]{0.25,0.44,0.63}{{#1}}}
    \newcommand{\StringTok}[1]{\textcolor[rgb]{0.25,0.44,0.63}{{#1}}}
    \newcommand{\CommentTok}[1]{\textcolor[rgb]{0.38,0.63,0.69}{\textit{{#1}}}}
    \newcommand{\OtherTok}[1]{\textcolor[rgb]{0.00,0.44,0.13}{{#1}}}
    \newcommand{\AlertTok}[1]{\textcolor[rgb]{1.00,0.00,0.00}{\textbf{{#1}}}}
    \newcommand{\FunctionTok}[1]{\textcolor[rgb]{0.02,0.16,0.49}{{#1}}}
    \newcommand{\RegionMarkerTok}[1]{{#1}}
    \newcommand{\ErrorTok}[1]{\textcolor[rgb]{1.00,0.00,0.00}{\textbf{{#1}}}}
    \newcommand{\NormalTok}[1]{{#1}}
    
    % Additional commands for more recent versions of Pandoc
    \newcommand{\ConstantTok}[1]{\textcolor[rgb]{0.53,0.00,0.00}{{#1}}}
    \newcommand{\SpecialCharTok}[1]{\textcolor[rgb]{0.25,0.44,0.63}{{#1}}}
    \newcommand{\VerbatimStringTok}[1]{\textcolor[rgb]{0.25,0.44,0.63}{{#1}}}
    \newcommand{\SpecialStringTok}[1]{\textcolor[rgb]{0.73,0.40,0.53}{{#1}}}
    \newcommand{\ImportTok}[1]{{#1}}
    \newcommand{\DocumentationTok}[1]{\textcolor[rgb]{0.73,0.13,0.13}{\textit{{#1}}}}
    \newcommand{\AnnotationTok}[1]{\textcolor[rgb]{0.38,0.63,0.69}{\textbf{\textit{{#1}}}}}
    \newcommand{\CommentVarTok}[1]{\textcolor[rgb]{0.38,0.63,0.69}{\textbf{\textit{{#1}}}}}
    \newcommand{\VariableTok}[1]{\textcolor[rgb]{0.10,0.09,0.49}{{#1}}}
    \newcommand{\ControlFlowTok}[1]{\textcolor[rgb]{0.00,0.44,0.13}{\textbf{{#1}}}}
    \newcommand{\OperatorTok}[1]{\textcolor[rgb]{0.40,0.40,0.40}{{#1}}}
    \newcommand{\BuiltInTok}[1]{{#1}}
    \newcommand{\ExtensionTok}[1]{{#1}}
    \newcommand{\PreprocessorTok}[1]{\textcolor[rgb]{0.74,0.48,0.00}{{#1}}}
    \newcommand{\AttributeTok}[1]{\textcolor[rgb]{0.49,0.56,0.16}{{#1}}}
    \newcommand{\InformationTok}[1]{\textcolor[rgb]{0.38,0.63,0.69}{\textbf{\textit{{#1}}}}}
    \newcommand{\WarningTok}[1]{\textcolor[rgb]{0.38,0.63,0.69}{\textbf{\textit{{#1}}}}}
    
    
    % Define a nice break command that doesn't care if a line doesn't already
    % exist.
    \def\br{\hspace*{\fill} \\* }
    % Math Jax compatibility definitions
    \def\gt{>}
    \def\lt{<}
    \let\Oldtex\TeX
    \let\Oldlatex\LaTeX
    \renewcommand{\TeX}{\textrm{\Oldtex}}
    \renewcommand{\LaTeX}{\textrm{\Oldlatex}}
    % Document parameters
    % Document title
    \title{CHE384T PS2: Lattice Sums, part 1}
    
    
    
    
    
% Pygments definitions
\makeatletter
\def\PY@reset{\let\PY@it=\relax \let\PY@bf=\relax%
    \let\PY@ul=\relax \let\PY@tc=\relax%
    \let\PY@bc=\relax \let\PY@ff=\relax}
\def\PY@tok#1{\csname PY@tok@#1\endcsname}
\def\PY@toks#1+{\ifx\relax#1\empty\else%
    \PY@tok{#1}\expandafter\PY@toks\fi}
\def\PY@do#1{\PY@bc{\PY@tc{\PY@ul{%
    \PY@it{\PY@bf{\PY@ff{#1}}}}}}}
\def\PY#1#2{\PY@reset\PY@toks#1+\relax+\PY@do{#2}}

\@namedef{PY@tok@w}{\def\PY@tc##1{\textcolor[rgb]{0.73,0.73,0.73}{##1}}}
\@namedef{PY@tok@c}{\let\PY@it=\textit\def\PY@tc##1{\textcolor[rgb]{0.24,0.48,0.48}{##1}}}
\@namedef{PY@tok@cp}{\def\PY@tc##1{\textcolor[rgb]{0.61,0.40,0.00}{##1}}}
\@namedef{PY@tok@k}{\let\PY@bf=\textbf\def\PY@tc##1{\textcolor[rgb]{0.00,0.50,0.00}{##1}}}
\@namedef{PY@tok@kp}{\def\PY@tc##1{\textcolor[rgb]{0.00,0.50,0.00}{##1}}}
\@namedef{PY@tok@kt}{\def\PY@tc##1{\textcolor[rgb]{0.69,0.00,0.25}{##1}}}
\@namedef{PY@tok@o}{\def\PY@tc##1{\textcolor[rgb]{0.40,0.40,0.40}{##1}}}
\@namedef{PY@tok@ow}{\let\PY@bf=\textbf\def\PY@tc##1{\textcolor[rgb]{0.67,0.13,1.00}{##1}}}
\@namedef{PY@tok@nb}{\def\PY@tc##1{\textcolor[rgb]{0.00,0.50,0.00}{##1}}}
\@namedef{PY@tok@nf}{\def\PY@tc##1{\textcolor[rgb]{0.00,0.00,1.00}{##1}}}
\@namedef{PY@tok@nc}{\let\PY@bf=\textbf\def\PY@tc##1{\textcolor[rgb]{0.00,0.00,1.00}{##1}}}
\@namedef{PY@tok@nn}{\let\PY@bf=\textbf\def\PY@tc##1{\textcolor[rgb]{0.00,0.00,1.00}{##1}}}
\@namedef{PY@tok@ne}{\let\PY@bf=\textbf\def\PY@tc##1{\textcolor[rgb]{0.80,0.25,0.22}{##1}}}
\@namedef{PY@tok@nv}{\def\PY@tc##1{\textcolor[rgb]{0.10,0.09,0.49}{##1}}}
\@namedef{PY@tok@no}{\def\PY@tc##1{\textcolor[rgb]{0.53,0.00,0.00}{##1}}}
\@namedef{PY@tok@nl}{\def\PY@tc##1{\textcolor[rgb]{0.46,0.46,0.00}{##1}}}
\@namedef{PY@tok@ni}{\let\PY@bf=\textbf\def\PY@tc##1{\textcolor[rgb]{0.44,0.44,0.44}{##1}}}
\@namedef{PY@tok@na}{\def\PY@tc##1{\textcolor[rgb]{0.41,0.47,0.13}{##1}}}
\@namedef{PY@tok@nt}{\let\PY@bf=\textbf\def\PY@tc##1{\textcolor[rgb]{0.00,0.50,0.00}{##1}}}
\@namedef{PY@tok@nd}{\def\PY@tc##1{\textcolor[rgb]{0.67,0.13,1.00}{##1}}}
\@namedef{PY@tok@s}{\def\PY@tc##1{\textcolor[rgb]{0.73,0.13,0.13}{##1}}}
\@namedef{PY@tok@sd}{\let\PY@it=\textit\def\PY@tc##1{\textcolor[rgb]{0.73,0.13,0.13}{##1}}}
\@namedef{PY@tok@si}{\let\PY@bf=\textbf\def\PY@tc##1{\textcolor[rgb]{0.64,0.35,0.47}{##1}}}
\@namedef{PY@tok@se}{\let\PY@bf=\textbf\def\PY@tc##1{\textcolor[rgb]{0.67,0.36,0.12}{##1}}}
\@namedef{PY@tok@sr}{\def\PY@tc##1{\textcolor[rgb]{0.64,0.35,0.47}{##1}}}
\@namedef{PY@tok@ss}{\def\PY@tc##1{\textcolor[rgb]{0.10,0.09,0.49}{##1}}}
\@namedef{PY@tok@sx}{\def\PY@tc##1{\textcolor[rgb]{0.00,0.50,0.00}{##1}}}
\@namedef{PY@tok@m}{\def\PY@tc##1{\textcolor[rgb]{0.40,0.40,0.40}{##1}}}
\@namedef{PY@tok@gh}{\let\PY@bf=\textbf\def\PY@tc##1{\textcolor[rgb]{0.00,0.00,0.50}{##1}}}
\@namedef{PY@tok@gu}{\let\PY@bf=\textbf\def\PY@tc##1{\textcolor[rgb]{0.50,0.00,0.50}{##1}}}
\@namedef{PY@tok@gd}{\def\PY@tc##1{\textcolor[rgb]{0.63,0.00,0.00}{##1}}}
\@namedef{PY@tok@gi}{\def\PY@tc##1{\textcolor[rgb]{0.00,0.52,0.00}{##1}}}
\@namedef{PY@tok@gr}{\def\PY@tc##1{\textcolor[rgb]{0.89,0.00,0.00}{##1}}}
\@namedef{PY@tok@ge}{\let\PY@it=\textit}
\@namedef{PY@tok@gs}{\let\PY@bf=\textbf}
\@namedef{PY@tok@gp}{\let\PY@bf=\textbf\def\PY@tc##1{\textcolor[rgb]{0.00,0.00,0.50}{##1}}}
\@namedef{PY@tok@go}{\def\PY@tc##1{\textcolor[rgb]{0.44,0.44,0.44}{##1}}}
\@namedef{PY@tok@gt}{\def\PY@tc##1{\textcolor[rgb]{0.00,0.27,0.87}{##1}}}
\@namedef{PY@tok@err}{\def\PY@bc##1{{\setlength{\fboxsep}{\string -\fboxrule}\fcolorbox[rgb]{1.00,0.00,0.00}{1,1,1}{\strut ##1}}}}
\@namedef{PY@tok@kc}{\let\PY@bf=\textbf\def\PY@tc##1{\textcolor[rgb]{0.00,0.50,0.00}{##1}}}
\@namedef{PY@tok@kd}{\let\PY@bf=\textbf\def\PY@tc##1{\textcolor[rgb]{0.00,0.50,0.00}{##1}}}
\@namedef{PY@tok@kn}{\let\PY@bf=\textbf\def\PY@tc##1{\textcolor[rgb]{0.00,0.50,0.00}{##1}}}
\@namedef{PY@tok@kr}{\let\PY@bf=\textbf\def\PY@tc##1{\textcolor[rgb]{0.00,0.50,0.00}{##1}}}
\@namedef{PY@tok@bp}{\def\PY@tc##1{\textcolor[rgb]{0.00,0.50,0.00}{##1}}}
\@namedef{PY@tok@fm}{\def\PY@tc##1{\textcolor[rgb]{0.00,0.00,1.00}{##1}}}
\@namedef{PY@tok@vc}{\def\PY@tc##1{\textcolor[rgb]{0.10,0.09,0.49}{##1}}}
\@namedef{PY@tok@vg}{\def\PY@tc##1{\textcolor[rgb]{0.10,0.09,0.49}{##1}}}
\@namedef{PY@tok@vi}{\def\PY@tc##1{\textcolor[rgb]{0.10,0.09,0.49}{##1}}}
\@namedef{PY@tok@vm}{\def\PY@tc##1{\textcolor[rgb]{0.10,0.09,0.49}{##1}}}
\@namedef{PY@tok@sa}{\def\PY@tc##1{\textcolor[rgb]{0.73,0.13,0.13}{##1}}}
\@namedef{PY@tok@sb}{\def\PY@tc##1{\textcolor[rgb]{0.73,0.13,0.13}{##1}}}
\@namedef{PY@tok@sc}{\def\PY@tc##1{\textcolor[rgb]{0.73,0.13,0.13}{##1}}}
\@namedef{PY@tok@dl}{\def\PY@tc##1{\textcolor[rgb]{0.73,0.13,0.13}{##1}}}
\@namedef{PY@tok@s2}{\def\PY@tc##1{\textcolor[rgb]{0.73,0.13,0.13}{##1}}}
\@namedef{PY@tok@sh}{\def\PY@tc##1{\textcolor[rgb]{0.73,0.13,0.13}{##1}}}
\@namedef{PY@tok@s1}{\def\PY@tc##1{\textcolor[rgb]{0.73,0.13,0.13}{##1}}}
\@namedef{PY@tok@mb}{\def\PY@tc##1{\textcolor[rgb]{0.40,0.40,0.40}{##1}}}
\@namedef{PY@tok@mf}{\def\PY@tc##1{\textcolor[rgb]{0.40,0.40,0.40}{##1}}}
\@namedef{PY@tok@mh}{\def\PY@tc##1{\textcolor[rgb]{0.40,0.40,0.40}{##1}}}
\@namedef{PY@tok@mi}{\def\PY@tc##1{\textcolor[rgb]{0.40,0.40,0.40}{##1}}}
\@namedef{PY@tok@il}{\def\PY@tc##1{\textcolor[rgb]{0.40,0.40,0.40}{##1}}}
\@namedef{PY@tok@mo}{\def\PY@tc##1{\textcolor[rgb]{0.40,0.40,0.40}{##1}}}
\@namedef{PY@tok@ch}{\let\PY@it=\textit\def\PY@tc##1{\textcolor[rgb]{0.24,0.48,0.48}{##1}}}
\@namedef{PY@tok@cm}{\let\PY@it=\textit\def\PY@tc##1{\textcolor[rgb]{0.24,0.48,0.48}{##1}}}
\@namedef{PY@tok@cpf}{\let\PY@it=\textit\def\PY@tc##1{\textcolor[rgb]{0.24,0.48,0.48}{##1}}}
\@namedef{PY@tok@c1}{\let\PY@it=\textit\def\PY@tc##1{\textcolor[rgb]{0.24,0.48,0.48}{##1}}}
\@namedef{PY@tok@cs}{\let\PY@it=\textit\def\PY@tc##1{\textcolor[rgb]{0.24,0.48,0.48}{##1}}}

\def\PYZbs{\char`\\}
\def\PYZus{\char`\_}
\def\PYZob{\char`\{}
\def\PYZcb{\char`\}}
\def\PYZca{\char`\^}
\def\PYZam{\char`\&}
\def\PYZlt{\char`\<}
\def\PYZgt{\char`\>}
\def\PYZsh{\char`\#}
\def\PYZpc{\char`\%}
\def\PYZdl{\char`\$}
\def\PYZhy{\char`\-}
\def\PYZsq{\char`\'}
\def\PYZdq{\char`\"}
\def\PYZti{\char`\~}
% for compatibility with earlier versions
\def\PYZat{@}
\def\PYZlb{[}
\def\PYZrb{]}
\makeatother


    % For linebreaks inside Verbatim environment from package fancyvrb. 
    \makeatletter
        \newbox\Wrappedcontinuationbox 
        \newbox\Wrappedvisiblespacebox 
        \newcommand*\Wrappedvisiblespace {\textcolor{red}{\textvisiblespace}} 
        \newcommand*\Wrappedcontinuationsymbol {\textcolor{red}{\llap{\tiny$\m@th\hookrightarrow$}}} 
        \newcommand*\Wrappedcontinuationindent {3ex } 
        \newcommand*\Wrappedafterbreak {\kern\Wrappedcontinuationindent\copy\Wrappedcontinuationbox} 
        % Take advantage of the already applied Pygments mark-up to insert 
        % potential linebreaks for TeX processing. 
        %        {, <, #, %, $, ' and ": go to next line. 
        %        _, }, ^, &, >, - and ~: stay at end of broken line. 
        % Use of \textquotesingle for straight quote. 
        \newcommand*\Wrappedbreaksatspecials {% 
            \def\PYGZus{\discretionary{\char`\_}{\Wrappedafterbreak}{\char`\_}}% 
            \def\PYGZob{\discretionary{}{\Wrappedafterbreak\char`\{}{\char`\{}}% 
            \def\PYGZcb{\discretionary{\char`\}}{\Wrappedafterbreak}{\char`\}}}% 
            \def\PYGZca{\discretionary{\char`\^}{\Wrappedafterbreak}{\char`\^}}% 
            \def\PYGZam{\discretionary{\char`\&}{\Wrappedafterbreak}{\char`\&}}% 
            \def\PYGZlt{\discretionary{}{\Wrappedafterbreak\char`\<}{\char`\<}}% 
            \def\PYGZgt{\discretionary{\char`\>}{\Wrappedafterbreak}{\char`\>}}% 
            \def\PYGZsh{\discretionary{}{\Wrappedafterbreak\char`\#}{\char`\#}}% 
            \def\PYGZpc{\discretionary{}{\Wrappedafterbreak\char`\%}{\char`\%}}% 
            \def\PYGZdl{\discretionary{}{\Wrappedafterbreak\char`\$}{\char`\$}}% 
            \def\PYGZhy{\discretionary{\char`\-}{\Wrappedafterbreak}{\char`\-}}% 
            \def\PYGZsq{\discretionary{}{\Wrappedafterbreak\textquotesingle}{\textquotesingle}}% 
            \def\PYGZdq{\discretionary{}{\Wrappedafterbreak\char`\"}{\char`\"}}% 
            \def\PYGZti{\discretionary{\char`\~}{\Wrappedafterbreak}{\char`\~}}% 
        } 
        % Some characters . , ; ? ! / are not pygmentized. 
        % This macro makes them "active" and they will insert potential linebreaks 
        \newcommand*\Wrappedbreaksatpunct {% 
            \lccode`\~`\.\lowercase{\def~}{\discretionary{\hbox{\char`\.}}{\Wrappedafterbreak}{\hbox{\char`\.}}}% 
            \lccode`\~`\,\lowercase{\def~}{\discretionary{\hbox{\char`\,}}{\Wrappedafterbreak}{\hbox{\char`\,}}}% 
            \lccode`\~`\;\lowercase{\def~}{\discretionary{\hbox{\char`\;}}{\Wrappedafterbreak}{\hbox{\char`\;}}}% 
            \lccode`\~`\:\lowercase{\def~}{\discretionary{\hbox{\char`\:}}{\Wrappedafterbreak}{\hbox{\char`\:}}}% 
            \lccode`\~`\?\lowercase{\def~}{\discretionary{\hbox{\char`\?}}{\Wrappedafterbreak}{\hbox{\char`\?}}}% 
            \lccode`\~`\!\lowercase{\def~}{\discretionary{\hbox{\char`\!}}{\Wrappedafterbreak}{\hbox{\char`\!}}}% 
            \lccode`\~`\/\lowercase{\def~}{\discretionary{\hbox{\char`\/}}{\Wrappedafterbreak}{\hbox{\char`\/}}}% 
            \catcode`\.\active
            \catcode`\,\active 
            \catcode`\;\active
            \catcode`\:\active
            \catcode`\?\active
            \catcode`\!\active
            \catcode`\/\active 
            \lccode`\~`\~ 	
        }
    \makeatother

    \let\OriginalVerbatim=\Verbatim
    \makeatletter
    \renewcommand{\Verbatim}[1][1]{%
        %\parskip\z@skip
        \sbox\Wrappedcontinuationbox {\Wrappedcontinuationsymbol}%
        \sbox\Wrappedvisiblespacebox {\FV@SetupFont\Wrappedvisiblespace}%
        \def\FancyVerbFormatLine ##1{\hsize\linewidth
            \vtop{\raggedright\hyphenpenalty\z@\exhyphenpenalty\z@
                \doublehyphendemerits\z@\finalhyphendemerits\z@
                \strut ##1\strut}%
        }%
        % If the linebreak is at a space, the latter will be displayed as visible
        % space at end of first line, and a continuation symbol starts next line.
        % Stretch/shrink are however usually zero for typewriter font.
        \def\FV@Space {%
            \nobreak\hskip\z@ plus\fontdimen3\font minus\fontdimen4\font
            \discretionary{\copy\Wrappedvisiblespacebox}{\Wrappedafterbreak}
            {\kern\fontdimen2\font}%
        }%
        
        % Allow breaks at special characters using \PYG... macros.
        \Wrappedbreaksatspecials
        % Breaks at punctuation characters . , ; ? ! and / need catcode=\active 	
        \OriginalVerbatim[#1,codes*=\Wrappedbreaksatpunct]%
    }
    \makeatother

    % Exact colors from NB
    \definecolor{incolor}{HTML}{303F9F}
    \definecolor{outcolor}{HTML}{D84315}
    \definecolor{cellborder}{HTML}{CFCFCF}
    \definecolor{cellbackground}{HTML}{F7F7F7}
    
    % prompt
    \makeatletter
    \newcommand{\boxspacing}{\kern\kvtcb@left@rule\kern\kvtcb@boxsep}
    \makeatother
    \newcommand{\prompt}[4]{
        {\ttfamily\llap{{\color{#2}[#3]:\hspace{3pt}#4}}\vspace{-\baselineskip}}
    }
    

    
    % Prevent overflowing lines due to hard-to-break entities
    \sloppy 
    % Setup hyperref package
    \hypersetup{
      breaklinks=true,  % so long urls are correctly broken across lines
      colorlinks=true,
      urlcolor=urlcolor,
      linkcolor=linkcolor,
      citecolor=citecolor,
      }
    % Slightly bigger margins than the latex defaults
    
    \geometry{verbose,tmargin=1in,bmargin=1in,lmargin=1in,rmargin=1in}
    
    

\begin{document}
    
    \maketitle
    
    

    
    \hypertarget{ps2-lattice-sums-and-simulation-of-finite-systems}{%
\section{PS2: Lattice Sums and Simulation of finite
systems}\label{ps2-lattice-sums-and-simulation-of-finite-systems}}

Pre-reqs: - jupyterlab-myst:
https://github.com/executablebooks/jupyterlab-myst

    \hypertarget{context-and-motivations}{%
\subsection{Context and Motivations}\label{context-and-motivations}}

Methods to model materials typically involve computing a system with
discrete objects, such as atoms, spins, or defects. The calculation
itself may involve computing a total energy, interaction energy, order
parameters, etc. Usually, we are faced with the challenge of modeling a
macroscopic system with many components or objects, which would be
computationally intractable to do in whole. Thus, we often use cutoffs
and boundary conditions to approximate (effectively) finite systems.

As an illustration of the basic components of simulating (effectively)
finite systems, we will perform lattice sums (of energy) on a
crystalline lattice. We will describe how to set up the simulation of a
cell that is replicated in space with periodic boundary conditions,
which is particularly useful when describing materials with translation
symmetry (e.g., crystals).

In materials modeling, summing over interactions is a common task. In
this problem set, we will demonstrate the calculation of summing the
interaction energy.

Consider a system of \(N\) atoms in a simulation cell in which the total
interaction energy may be written as

\begin{equation}
U = \frac{1}{2} \sum_{i=1}^N \sum_{j=1}^N \phi_{ij}(r_{ij})
\end{equation}

where \(i=j\) terms are omitted and \(\phi_{ij}\) is a pair potential
that depends only on the distance between atoms

\begin{equation}
r_{ij} = [(x_j - x_i)^2 + (y_j - y_i)^2 +(z_j - z_i)^2]^{\frac{1}{2}}
\end{equation}

As \(N\) increases, so does the computation cost to perform the lattice
sum. Thus, it is common practice to assume that at some separation, the
interatomic potential is small enough to be negligible. The distance at
which the interatomic potential is considered negligible is the cutoff
distance, \(r_c\). Note that slowly decaying potentials that go as
\(1/r^n\) where $n \leq 3 $ require more sophisticated methods beyond
using a simple cutoff distance.

The most straightforward method to include a cutoff is to explicitly
test the distance between two atoms and not include interactions for
\(r_{ij} > r_c\). The task is then to sum over only the neighbors that
are within the cutoff distance.

Many systems of interest are crystalline, meaning the material has
translational symmetry and can be described using a unit cell with
periodic boundary conditions. A central simulation cell is chosen and
effectively replicated to fill space. The possible symmetries of a unit
cell are described with the Bravais lattices. In this exercise, we will
assume a cubic unit cell.

Consider an atom \(i\) at position $\vec{r}_i$. In a periodic cell,
the replicas of atom \(i\) are located at $\vec{R} + \vec{r}_i$ where
$\vec{R}$ is a lattice vector of the simulation cell. The energy then
becomes

\begin{equation}
U = \frac{1}{2}\sum_{\vec{R}} \sum_{i=1}^N \sum_{j=1}^N \phi_{ij} (|\vec{R}+\vec{r}_j-\vec{r}_i|)
\end{equation}

where we sum over the lattice vectors \(\vec{R}\) and omit terms
relating to \(i=j\) and \(\vec{R} = 0\). We will use periodic boundary
conditions to mimic the properties of the larger (crystalline) system.

    \hypertarget{implementation-of-lattice-sums}{%
\subsection{Implementation of Lattice
Sums}\label{implementation-of-lattice-sums}}

    \hypertarget{generating-an-fcc-lattice}{%
\subsubsection{Generating an FCC
lattice}\label{generating-an-fcc-lattice}}

Let us generate a simulation over an FCC unit cell in the conventional
cubic Bravais lattice. In the conventional FCC unit cell, there are four
atoms with position vectors (written in fractional coordinates in terms
of the lattice parameter \(a\))

\begin{equation}
\begin{aligned}
\vec{r}_1 = (0,0,0) \\
\vec{r}_2 =(\frac{1}{2},\frac{1}{2},0) \\
\vec{r}_3 =(\frac{1}{2},0,\frac{1}{2}) \\
\vec{r}_4 = (0,\frac{1}{2},\frac{1}{2})
\end{aligned}
\end{equation}

For a simulation cell with \(n_c\) FCC cells in each direction, we can
repeat the unit cell by the lattice vector \(\vec{R}\)

\begin{equation}
\vec{R} = (i,j,k) \text{ for } i = 0,1,2,... n_c-1, j = 0,1,2,...n_c-1, \text{ and } k = 0,1,2,...n_c-1
\end{equation}

to each of the position vectors. We then divide each of the four
position vectors by \(n_c\) so that they are given a new fractional
coordinates. For example, suppose we had a one-dimensional structure
with 1 atom per cell at a position with \(x = 0.5\). To create a
simulation cell with 2 of these cells, we would create two positions at
\(x = 0.25\) and \(x = 0.75\). As another example, if \(n_c = 2\), we
create a simulation cell with the atoms in an FCC structure but with a
\(4 \times 8 \,\text{unit cells}= 32\) atoms per unit cell.

We can construct a \(n_c \times n_c \times n_c\) FCC unit cells in
Python using arrays. We first introduce a vector \(r\) containing the
\(n=4\) positions of the FCC atomic basis,

\begin{verbatim}
r = [0,0,0; 0.5,0.5,0; 0.5,0,0.5, 0,0.5,0.5]
\end{verbatim}

We then add lattice vectors to the \(x\), \(y\), and \(z\) coordinates
of the atoms in the cell. For a cubic material, the lattice vector has
the form \(\vec{R} = (k,l,m)a\), where \(a\) is the lattice length.

We loop over the indices of \(\vec{R}\) and divide by \(n_c\) to ensure
that the atomic positions are in the correct fractional coordinates in
the simulation cell.

In the following code, we generate the (fractional) atomic coordinates
of an FCC supercell (omitting the lattice parameter). The output of the
following code is an array of fractional atomic coordinates of length
\(4 \times n_c \times n_c \times n_c\).

    \begin{tcolorbox}[breakable, size=fbox, boxrule=1pt, pad at break*=1mm,colback=cellbackground, colframe=cellborder]
\prompt{In}{incolor}{5}{\boxspacing}
\begin{Verbatim}[commandchars=\\\{\}]
\PY{c+c1}{\PYZsh{}\PYZsh{} Code for generating an FCC supercell}
\PY{k+kn}{import} \PY{n+nn}{numpy} \PY{k}{as} \PY{n+nn}{np}

\PY{k}{def} \PY{n+nf}{fccmke}\PY{p}{(}\PY{n}{nc}\PY{p}{)}\PY{p}{:}
    \PY{l+s+sd}{\PYZdq{}\PYZdq{}\PYZdq{} Generate fractional coordinates of atoms in FCC supercell}
\PY{l+s+sd}{       Input:}
\PY{l+s+sd}{           nc (integer): supercell of nc x nc x nc dimensions}
\PY{l+s+sd}{       Output:}
\PY{l+s+sd}{           s (array): fractional atomic coordinates of supercell}
\PY{l+s+sd}{    \PYZdq{}\PYZdq{}\PYZdq{}} 

    \PY{n}{natoms} \PY{o}{=} \PY{l+m+mi}{4} 
    \PY{n}{r} \PY{o}{=} \PY{n}{np}\PY{o}{.}\PY{n}{array}\PY{p}{(}\PY{p}{[}\PY{p}{[}\PY{l+m+mi}{0}\PY{p}{,} \PY{l+m+mi}{0}\PY{p}{,} \PY{l+m+mi}{0}\PY{p}{]}\PY{p}{,} \PY{p}{[}\PY{l+m+mf}{0.5}\PY{p}{,} \PY{l+m+mf}{0.5}\PY{p}{,} \PY{l+m+mi}{0}\PY{p}{]}\PY{p}{,} \PY{p}{[}\PY{l+m+mi}{0}\PY{p}{,} \PY{l+m+mf}{0.5}\PY{p}{,} \PY{l+m+mf}{0.5}\PY{p}{]}\PY{p}{,} \PY{p}{[}\PY{l+m+mf}{0.5}\PY{p}{,} \PY{l+m+mi}{0}\PY{p}{,} \PY{l+m+mf}{0.5}\PY{p}{]}\PY{p}{]}\PY{p}{)}
    \PY{n}{i1} \PY{o}{=} \PY{l+m+mi}{0} 
    \PY{n}{s} \PY{o}{=} \PY{n}{np}\PY{o}{.}\PY{n}{zeros}\PY{p}{(}\PY{p}{(}\PY{n}{natoms} \PY{o}{*} \PY{n}{nc}\PY{o}{*}\PY{o}{*}\PY{l+m+mi}{3}\PY{p}{,} \PY{l+m+mi}{3}\PY{p}{)}\PY{p}{)} 

    \PY{c+c1}{\PYZsh{} in fractional coordinates}
    \PY{k}{for} \PY{n}{k} \PY{o+ow}{in} \PY{n+nb}{range}\PY{p}{(}\PY{l+m+mi}{1}\PY{p}{,} \PY{n}{nc} \PY{o}{+} \PY{l+m+mi}{1}\PY{p}{)}\PY{p}{:} 
        \PY{k}{for} \PY{n}{l} \PY{o+ow}{in} \PY{n+nb}{range}\PY{p}{(}\PY{l+m+mi}{1}\PY{p}{,} \PY{n}{nc} \PY{o}{+} \PY{l+m+mi}{1}\PY{p}{)}\PY{p}{:} 
            \PY{k}{for} \PY{n}{m} \PY{o+ow}{in} \PY{n+nb}{range}\PY{p}{(}\PY{l+m+mi}{1}\PY{p}{,} \PY{n}{nc} \PY{o}{+} \PY{l+m+mi}{1}\PY{p}{)}\PY{p}{:} 
                \PY{k}{for} \PY{n}{i} \PY{o+ow}{in} \PY{n+nb}{range}\PY{p}{(}\PY{n}{natoms}\PY{p}{)}\PY{p}{:}
                    \PY{n}{s}\PY{p}{[}\PY{n}{i1}\PY{p}{,} \PY{l+m+mi}{0}\PY{p}{]} \PY{o}{=} \PY{p}{(}\PY{n}{r}\PY{p}{[}\PY{n}{i}\PY{p}{,} \PY{l+m+mi}{0}\PY{p}{]} \PY{o}{+} \PY{n}{k} \PY{o}{\PYZhy{}} \PY{l+m+mi}{1}\PY{p}{)} \PY{o}{/} \PY{n}{nc}
                    \PY{n}{s}\PY{p}{[}\PY{n}{i1}\PY{p}{,} \PY{l+m+mi}{1}\PY{p}{]} \PY{o}{=} \PY{p}{(}\PY{n}{r}\PY{p}{[}\PY{n}{i}\PY{p}{,} \PY{l+m+mi}{1}\PY{p}{]} \PY{o}{+} \PY{n}{l} \PY{o}{\PYZhy{}} \PY{l+m+mi}{1}\PY{p}{)} \PY{o}{/} \PY{n}{nc}
                    \PY{n}{s}\PY{p}{[}\PY{n}{i1}\PY{p}{,} \PY{l+m+mi}{2}\PY{p}{]} \PY{o}{=} \PY{p}{(}\PY{n}{r}\PY{p}{[}\PY{n}{i}\PY{p}{,} \PY{l+m+mi}{2}\PY{p}{]} \PY{o}{+} \PY{n}{m} \PY{o}{\PYZhy{}} \PY{l+m+mi}{1}\PY{p}{)} \PY{o}{/} \PY{n}{nc}
                    \PY{n}{i1} \PY{o}{+}\PY{o}{=} \PY{l+m+mi}{1}
    \PY{k}{return} \PY{n}{s}

\PY{c+c1}{\PYZsh{} Example usage:}
\PY{n}{nc} \PY{o}{=} \PY{l+m+mi}{3}
\PY{n}{s} \PY{o}{=} \PY{n}{fccmke}\PY{p}{(}\PY{n}{nc}\PY{p}{)}
\PY{n+nb}{print}\PY{p}{(}\PY{n}{s}\PY{p}{)}
\end{Verbatim}
\end{tcolorbox}

    \begin{Verbatim}[commandchars=\\\{\}]
[[0.         0.         0.        ]
 [0.16666667 0.16666667 0.        ]
 [0.         0.16666667 0.16666667]
 [0.16666667 0.         0.16666667]
 [0.         0.         0.33333333]
 [0.16666667 0.16666667 0.33333333]
 [0.         0.16666667 0.5       ]
 [0.16666667 0.         0.5       ]
 [0.         0.         0.66666667]
 [0.16666667 0.16666667 0.66666667]
 [0.         0.16666667 0.83333333]
 [0.16666667 0.         0.83333333]
 [0.         0.33333333 0.        ]
 [0.16666667 0.5        0.        ]
 [0.         0.5        0.16666667]
 [0.16666667 0.33333333 0.16666667]
 [0.         0.33333333 0.33333333]
 [0.16666667 0.5        0.33333333]
 [0.         0.5        0.5       ]
 [0.16666667 0.33333333 0.5       ]
 [0.         0.33333333 0.66666667]
 [0.16666667 0.5        0.66666667]
 [0.         0.5        0.83333333]
 [0.16666667 0.33333333 0.83333333]
 [0.         0.66666667 0.        ]
 [0.16666667 0.83333333 0.        ]
 [0.         0.83333333 0.16666667]
 [0.16666667 0.66666667 0.16666667]
 [0.         0.66666667 0.33333333]
 [0.16666667 0.83333333 0.33333333]
 [0.         0.83333333 0.5       ]
 [0.16666667 0.66666667 0.5       ]
 [0.         0.66666667 0.66666667]
 [0.16666667 0.83333333 0.66666667]
 [0.         0.83333333 0.83333333]
 [0.16666667 0.66666667 0.83333333]
 [0.33333333 0.         0.        ]
 [0.5        0.16666667 0.        ]
 [0.33333333 0.16666667 0.16666667]
 [0.5        0.         0.16666667]
 [0.33333333 0.         0.33333333]
 [0.5        0.16666667 0.33333333]
 [0.33333333 0.16666667 0.5       ]
 [0.5        0.         0.5       ]
 [0.33333333 0.         0.66666667]
 [0.5        0.16666667 0.66666667]
 [0.33333333 0.16666667 0.83333333]
 [0.5        0.         0.83333333]
 [0.33333333 0.33333333 0.        ]
 [0.5        0.5        0.        ]
 [0.33333333 0.5        0.16666667]
 [0.5        0.33333333 0.16666667]
 [0.33333333 0.33333333 0.33333333]
 [0.5        0.5        0.33333333]
 [0.33333333 0.5        0.5       ]
 [0.5        0.33333333 0.5       ]
 [0.33333333 0.33333333 0.66666667]
 [0.5        0.5        0.66666667]
 [0.33333333 0.5        0.83333333]
 [0.5        0.33333333 0.83333333]
 [0.33333333 0.66666667 0.        ]
 [0.5        0.83333333 0.        ]
 [0.33333333 0.83333333 0.16666667]
 [0.5        0.66666667 0.16666667]
 [0.33333333 0.66666667 0.33333333]
 [0.5        0.83333333 0.33333333]
 [0.33333333 0.83333333 0.5       ]
 [0.5        0.66666667 0.5       ]
 [0.33333333 0.66666667 0.66666667]
 [0.5        0.83333333 0.66666667]
 [0.33333333 0.83333333 0.83333333]
 [0.5        0.66666667 0.83333333]
 [0.66666667 0.         0.        ]
 [0.83333333 0.16666667 0.        ]
 [0.66666667 0.16666667 0.16666667]
 [0.83333333 0.         0.16666667]
 [0.66666667 0.         0.33333333]
 [0.83333333 0.16666667 0.33333333]
 [0.66666667 0.16666667 0.5       ]
 [0.83333333 0.         0.5       ]
 [0.66666667 0.         0.66666667]
 [0.83333333 0.16666667 0.66666667]
 [0.66666667 0.16666667 0.83333333]
 [0.83333333 0.         0.83333333]
 [0.66666667 0.33333333 0.        ]
 [0.83333333 0.5        0.        ]
 [0.66666667 0.5        0.16666667]
 [0.83333333 0.33333333 0.16666667]
 [0.66666667 0.33333333 0.33333333]
 [0.83333333 0.5        0.33333333]
 [0.66666667 0.5        0.5       ]
 [0.83333333 0.33333333 0.5       ]
 [0.66666667 0.33333333 0.66666667]
 [0.83333333 0.5        0.66666667]
 [0.66666667 0.5        0.83333333]
 [0.83333333 0.33333333 0.83333333]
 [0.66666667 0.66666667 0.        ]
 [0.83333333 0.83333333 0.        ]
 [0.66666667 0.83333333 0.16666667]
 [0.83333333 0.66666667 0.16666667]
 [0.66666667 0.66666667 0.33333333]
 [0.83333333 0.83333333 0.33333333]
 [0.66666667 0.83333333 0.5       ]
 [0.83333333 0.66666667 0.5       ]
 [0.66666667 0.66666667 0.66666667]
 [0.83333333 0.83333333 0.66666667]
 [0.66666667 0.83333333 0.83333333]
 [0.83333333 0.66666667 0.83333333]]
    \end{Verbatim}

    \hypertarget{a-simple-lattice-sum}{%
\subsection{A simple lattice sum}\label{a-simple-lattice-sum}}

Let us consider the simple sum

\(U = \frac{1}{2} \sum_{i=1}^N \sum_{j=1}^N \phi_{ij}(r_{ij})\)

A simple model for the interatomic (pair-wise) potential is to assume
the potential is a function of only the distance between the atoms. For
now, let us assume the potential is the well-known Lennard-Jones
potential of the form

\(\phi(r) = 4\varepsilon\Big[ \Big(\frac{\sigma}{r}\Big)^{12} - \Big(\frac{\sigma}{r}\Big)^{6}\Big]\)

where we introduce the parameters \(\sigma\) and \(\varepsilon\) that
depend on the identity of the atom; \(\sigma\) is a the distance at
which the potential is zero, i.e., \(\phi(\sigma) = 0\) and
\(\varepsilon\) is the absolute value of the minimum of the potential.
For convenience, in the following code examples, we set
\(\sigma = \varepsilon = 1\).

Consider a simple lattice sum over N atoms in a cubic cell without
periodic boundary conditions. We assume that we have a vector \(s\)
containing the positions of the atoms in fractional coordinates (which
must be multiplied by the cube side \(a\)). Let us sum over the
Lennard-Jones interaction for an FCC supercell, as shown in the code
below.

    \begin{tcolorbox}[breakable, size=fbox, boxrule=1pt, pad at break*=1mm,colback=cellbackground, colframe=cellborder]
\prompt{In}{incolor}{ }{\boxspacing}
\begin{Verbatim}[commandchars=\\\{\}]
\PY{k+kn}{import} \PY{n+nn}{numpy} \PY{k}{as} \PY{n+nn}{np}
\PY{k+kn}{import} \PY{n+nn}{time} \PY{c+c1}{\PYZsh{} simple timer}
\end{Verbatim}
\end{tcolorbox}

    \begin{tcolorbox}[breakable, size=fbox, boxrule=1pt, pad at break*=1mm,colback=cellbackground, colframe=cellborder]
\prompt{In}{incolor}{7}{\boxspacing}
\begin{Verbatim}[commandchars=\\\{\}]
\PY{k}{def} \PY{n+nf}{lat\PYZus{}sum1}\PY{p}{(}\PY{n}{a}\PY{p}{,} \PY{n}{n}\PY{p}{,} \PY{n}{s}\PY{p}{)}\PY{p}{:} 
    \PY{l+s+sd}{\PYZdq{}\PYZdq{}\PYZdq{} Naive implementation of lattice sum with Lennard\PYZhy{}Jones potential}
\PY{l+s+sd}{        }
\PY{l+s+sd}{        Input:}
\PY{l+s+sd}{          a (float): cell length}
\PY{l+s+sd}{          n (integer): number of atoms in simulation cell}
\PY{l+s+sd}{          s (array): fractional coordinates of atom positions}
\PY{l+s+sd}{        Output:}
\PY{l+s+sd}{          ucell (float): total energy of simulation cell}
\PY{l+s+sd}{    \PYZdq{}\PYZdq{}\PYZdq{}}
    \PY{n}{start} \PY{o}{=} \PY{n}{time}\PY{o}{.}\PY{n}{process\PYZus{}time}\PY{p}{(}\PY{p}{)}

    \PY{n}{ucell} \PY{o}{=} \PY{l+m+mi}{0} 

    \PY{c+c1}{\PYZsh{} sum over all atoms and divide by two }
    \PY{k}{for} \PY{n}{i} \PY{o+ow}{in} \PY{n+nb}{range}\PY{p}{(}\PY{n}{n}\PY{p}{)}\PY{p}{:}
        \PY{k}{for} \PY{n}{j} \PY{o+ow}{in} \PY{n+nb}{range}\PY{p}{(}\PY{n}{n}\PY{p}{)}\PY{p}{:}
            \PY{n}{xij} \PY{o}{=} \PY{n}{s}\PY{p}{[}\PY{n}{j}\PY{p}{,} \PY{l+m+mi}{0}\PY{p}{]} \PY{o}{\PYZhy{}} \PY{n}{s}\PY{p}{[}\PY{n}{i}\PY{p}{,} \PY{l+m+mi}{0}\PY{p}{]}
            \PY{n}{yij} \PY{o}{=} \PY{n}{s}\PY{p}{[}\PY{n}{j}\PY{p}{,} \PY{l+m+mi}{1}\PY{p}{]} \PY{o}{\PYZhy{}} \PY{n}{s}\PY{p}{[}\PY{n}{i}\PY{p}{,} \PY{l+m+mi}{1}\PY{p}{]}
            \PY{n}{zij} \PY{o}{=} \PY{n}{s}\PY{p}{[}\PY{n}{j}\PY{p}{,} \PY{l+m+mi}{2}\PY{p}{]} \PY{o}{\PYZhy{}} \PY{n}{s}\PY{p}{[}\PY{n}{i}\PY{p}{,} \PY{l+m+mi}{2}\PY{p}{]}
            \PY{n}{dist} \PY{o}{=} \PY{n}{a} \PY{o}{*} \PY{n}{np}\PY{o}{.}\PY{n}{sqrt}\PY{p}{(}\PY{n}{xij}\PY{o}{*}\PY{o}{*}\PY{l+m+mi}{2} \PY{o}{+} \PY{n}{yij}\PY{o}{*}\PY{o}{*}\PY{l+m+mi}{2} \PY{o}{+} \PY{n}{zij}\PY{o}{*}\PY{o}{*}\PY{l+m+mi}{2}\PY{p}{)}
            \PY{k}{if} \PY{n}{dist} \PY{o}{\PYZgt{}} \PY{l+m+mi}{0}\PY{p}{:}
                \PY{n}{phi} \PY{o}{=} \PY{l+m+mi}{4} \PY{o}{*} \PY{p}{(}\PY{l+m+mi}{1} \PY{o}{/} \PY{n}{dist}\PY{o}{*}\PY{o}{*}\PY{l+m+mi}{12} \PY{o}{\PYZhy{}} \PY{l+m+mi}{1} \PY{o}{/} \PY{n}{dist}\PY{o}{*}\PY{o}{*}\PY{l+m+mi}{6}\PY{p}{)}
            \PY{k}{else}\PY{p}{:}
                \PY{n}{phi} \PY{o}{=} \PY{l+m+mi}{0} 
            \PY{n}{ucell} \PY{o}{=} \PY{n}{ucell} \PY{o}{+} \PY{n}{phi} 
    \PY{n}{ucell} \PY{o}{=} \PY{n}{ucell} \PY{o}{/} \PY{p}{(}\PY{l+m+mi}{2} \PY{o}{*} \PY{n}{n}\PY{p}{)}
   
    \PY{n}{end} \PY{o}{=} \PY{n}{time}\PY{o}{.}\PY{n}{process\PYZus{}time}\PY{p}{(}\PY{p}{)} \PY{o}{\PYZhy{}} \PY{n}{start}
    \PY{k}{return} \PY{n}{ucell}\PY{p}{,} \PY{n}{end} 
\end{Verbatim}
\end{tcolorbox}

    This code is quite inefficient. \texttt{lat\_sum.py} compute every
pairwise interaction twice, and so must divide by 2. A simple
improvement to our lattice sum code is to avoid overcounting with the
equivalent expression.

\(U = \sum_{i=1}^N \sum_{j=i+1}^N \phi_{ij}(r_{ij})\)

An implementation of the above is shown in the code below.

    \begin{tcolorbox}[breakable, size=fbox, boxrule=1pt, pad at break*=1mm,colback=cellbackground, colframe=cellborder]
\prompt{In}{incolor}{23}{\boxspacing}
\begin{Verbatim}[commandchars=\\\{\}]
\PY{k}{def} \PY{n+nf}{lat\PYZus{}sum2}\PY{p}{(}\PY{n}{a}\PY{p}{,} \PY{n}{n}\PY{p}{,} \PY{n}{s}\PY{p}{)}\PY{p}{:} 
    \PY{l+s+sd}{\PYZdq{}\PYZdq{}\PYZdq{} Avoid overcounting in lattice sums}

\PY{l+s+sd}{        Input:}
\PY{l+s+sd}{          a (float): cell length}
\PY{l+s+sd}{          n (integer): number of atoms in simulation cell}
\PY{l+s+sd}{          s (array): fractional coordinates of atom positions}
\PY{l+s+sd}{        Output:}
\PY{l+s+sd}{          ucell (float): total energy of simulation cell}
\PY{l+s+sd}{    \PYZdq{}\PYZdq{}\PYZdq{}}
    \PY{n}{start} \PY{o}{=} \PY{n}{time}\PY{o}{.}\PY{n}{process\PYZus{}time}\PY{p}{(}\PY{p}{)}

    \PY{n}{ucell} \PY{o}{=} \PY{l+m+mi}{0} 
    \PY{c+c1}{\PYZsh{} to avoid double counting, we change the indices over the loops}
    \PY{c+c1}{\PYZsh{} i will never equal j, so the if statement can be removed}
    \PY{c+c1}{\PYZsh{} no more factor of two!}
    \PY{k}{for} \PY{n}{i} \PY{o+ow}{in} \PY{n+nb}{range}\PY{p}{(}\PY{n}{n} \PY{o}{\PYZhy{}} \PY{l+m+mi}{1}\PY{p}{)}\PY{p}{:} 
        \PY{k}{for} \PY{n}{j} \PY{o+ow}{in} \PY{n+nb}{range}\PY{p}{(}\PY{n}{i} \PY{o}{+} \PY{l+m+mi}{1}\PY{p}{,} \PY{n}{n}\PY{p}{)}\PY{p}{:} 
            \PY{n}{xij} \PY{o}{=} \PY{n}{s}\PY{p}{[}\PY{n}{j}\PY{p}{,} \PY{l+m+mi}{0}\PY{p}{]} \PY{o}{\PYZhy{}} \PY{n}{s}\PY{p}{[}\PY{n}{i}\PY{p}{,} \PY{l+m+mi}{0}\PY{p}{]}
            \PY{n}{yij} \PY{o}{=} \PY{n}{s}\PY{p}{[}\PY{n}{j}\PY{p}{,} \PY{l+m+mi}{1}\PY{p}{]} \PY{o}{\PYZhy{}} \PY{n}{s}\PY{p}{[}\PY{n}{i}\PY{p}{,} \PY{l+m+mi}{1}\PY{p}{]}
            \PY{n}{zij} \PY{o}{=} \PY{n}{s}\PY{p}{[}\PY{n}{j}\PY{p}{,} \PY{l+m+mi}{2}\PY{p}{]} \PY{o}{\PYZhy{}} \PY{n}{s}\PY{p}{[}\PY{n}{i}\PY{p}{,} \PY{l+m+mi}{2}\PY{p}{]}
            \PY{n}{dist} \PY{o}{=} \PY{n}{a} \PY{o}{*} \PY{n}{np}\PY{o}{.}\PY{n}{sqrt}\PY{p}{(}\PY{n}{xij}\PY{o}{*}\PY{o}{*}\PY{l+m+mi}{2} \PY{o}{+} \PY{n}{yij}\PY{o}{*}\PY{o}{*}\PY{l+m+mi}{2} \PY{o}{+} \PY{n}{zij}\PY{o}{*}\PY{o}{*}\PY{l+m+mi}{2}\PY{p}{)}
            \PY{n}{phi} \PY{o}{=} \PY{l+m+mi}{4} \PY{o}{*} \PY{p}{(}\PY{l+m+mi}{1} \PY{o}{/} \PY{n}{dist}\PY{o}{*}\PY{o}{*}\PY{l+m+mi}{12} \PY{o}{\PYZhy{}} \PY{l+m+mi}{1} \PY{o}{/} \PY{n}{dist}\PY{o}{*}\PY{o}{*}\PY{l+m+mi}{6}\PY{p}{)}
            \PY{n}{ucell} \PY{o}{=} \PY{n}{ucell} \PY{o}{+} \PY{n}{phi}
    \PY{n}{ucell} \PY{o}{=} \PY{n}{ucell} \PY{o}{/} \PY{n}{n}

    \PY{n}{end} \PY{o}{=} \PY{n}{time}\PY{o}{.}\PY{n}{process\PYZus{}time}\PY{p}{(}\PY{p}{)} \PY{o}{\PYZhy{}} \PY{n}{start}
    \PY{k}{return} \PY{n}{ucell}\PY{p}{,} \PY{n}{end}
\end{Verbatim}
\end{tcolorbox}

    \hypertarget{cutoffs}{%
\subsubsection{Cutoffs}\label{cutoffs}}

Often to speed up calculations, we choose to truncate the interatomic
potential at some prescribed distance, referred to as the cutoff
distance \(r_c\).

We can implement a cutoff distance by including an \texttt{if} statement
to test if the distances are less than the cutoff. This is implemented
in the code below.

    \begin{tcolorbox}[breakable, size=fbox, boxrule=1pt, pad at break*=1mm,colback=cellbackground, colframe=cellborder]
\prompt{In}{incolor}{27}{\boxspacing}
\begin{Verbatim}[commandchars=\\\{\}]
\PY{k}{def} \PY{n+nf}{lat\PYZus{}sum3}\PY{p}{(}\PY{n}{a}\PY{p}{,} \PY{n}{n}\PY{p}{,} \PY{n}{rc}\PY{p}{,} \PY{n}{s}\PY{p}{)}\PY{p}{:}
    \PY{l+s+sd}{\PYZdq{}\PYZdq{}\PYZdq{} Inclusion of cutoff distance}

\PY{l+s+sd}{        Input:}
\PY{l+s+sd}{          a (float): cell length}
\PY{l+s+sd}{          n (integer): number of atoms in simulation cell}
\PY{l+s+sd}{          rc (float): cutoff distance}
\PY{l+s+sd}{          s (array): fractional coordinates of atom positions}
\PY{l+s+sd}{        Output:}
\PY{l+s+sd}{          ucell (float): total energy of simulation cell}
\PY{l+s+sd}{    \PYZdq{}\PYZdq{}\PYZdq{}}
    \PY{n}{start} \PY{o}{=} \PY{n}{time}\PY{o}{.}\PY{n}{process\PYZus{}time}\PY{p}{(}\PY{p}{)}

    \PY{n}{ucell} \PY{o}{=} \PY{l+m+mi}{0}
    \PY{k}{for} \PY{n}{i} \PY{o+ow}{in} \PY{n+nb}{range}\PY{p}{(}\PY{n}{n} \PY{o}{\PYZhy{}} \PY{l+m+mi}{1}\PY{p}{)}\PY{p}{:}
        \PY{k}{for} \PY{n}{j} \PY{o+ow}{in} \PY{n+nb}{range}\PY{p}{(}\PY{n}{i} \PY{o}{+} \PY{l+m+mi}{1}\PY{p}{,} \PY{n}{n}\PY{p}{)}\PY{p}{:}
            \PY{n}{xij} \PY{o}{=} \PY{n}{s}\PY{p}{[}\PY{n}{j}\PY{p}{,} \PY{l+m+mi}{0}\PY{p}{]} \PY{o}{\PYZhy{}} \PY{n}{s}\PY{p}{[}\PY{n}{i}\PY{p}{,} \PY{l+m+mi}{0}\PY{p}{]}
            \PY{n}{yij} \PY{o}{=} \PY{n}{s}\PY{p}{[}\PY{n}{j}\PY{p}{,} \PY{l+m+mi}{1}\PY{p}{]} \PY{o}{\PYZhy{}} \PY{n}{s}\PY{p}{[}\PY{n}{i}\PY{p}{,} \PY{l+m+mi}{1}\PY{p}{]}
            \PY{n}{zij} \PY{o}{=} \PY{n}{s}\PY{p}{[}\PY{n}{j}\PY{p}{,} \PY{l+m+mi}{2}\PY{p}{]} \PY{o}{\PYZhy{}} \PY{n}{s}\PY{p}{[}\PY{n}{i}\PY{p}{,} \PY{l+m+mi}{2}\PY{p}{]}
            \PY{n}{dist} \PY{o}{=} \PY{n}{a} \PY{o}{*} \PY{n}{np}\PY{o}{.}\PY{n}{sqrt}\PY{p}{(}\PY{n}{xij}\PY{o}{*}\PY{o}{*}\PY{l+m+mi}{2} \PY{o}{+} \PY{n}{yij}\PY{o}{*}\PY{o}{*}\PY{l+m+mi}{2} \PY{o}{+} \PY{n}{zij}\PY{o}{*}\PY{o}{*}\PY{l+m+mi}{2}\PY{p}{)}
            \PY{k}{if} \PY{n}{dist} \PY{o}{\PYZlt{}}\PY{o}{=} \PY{n}{rc}\PY{p}{:}
                \PY{n}{phi} \PY{o}{=} \PY{l+m+mi}{4} \PY{o}{*} \PY{p}{(}\PY{l+m+mi}{1} \PY{o}{/} \PY{n}{dist}\PY{o}{*}\PY{o}{*}\PY{l+m+mi}{12} \PY{o}{\PYZhy{}} \PY{l+m+mi}{1} \PY{o}{/} \PY{n}{dist}\PY{o}{*}\PY{o}{*}\PY{l+m+mi}{6}\PY{p}{)}
            \PY{k}{else}\PY{p}{:}
                \PY{n}{phi} \PY{o}{=} \PY{l+m+mi}{0}
            \PY{n}{ucell} \PY{o}{=} \PY{n}{ucell} \PY{o}{+} \PY{n}{phi}
    \PY{n}{ucell} \PY{o}{=} \PY{n}{ucell} \PY{o}{/} \PY{n}{n}

    \PY{n}{end} \PY{o}{=} \PY{n}{time}\PY{o}{.}\PY{n}{process\PYZus{}time}\PY{p}{(}\PY{p}{)} \PY{o}{\PYZhy{}} \PY{n}{start}
    \PY{k}{return} \PY{n}{ucell}\PY{p}{,} \PY{n}{end}
\end{Verbatim}
\end{tcolorbox}

    \hypertarget{periodic-boundary-conditions}{%
\subsubsection{Periodic Boundary
Conditions}\label{periodic-boundary-conditions}}

To mimic an infinite system, we will use periodic boundary conditions in
which the central simulation cell is replicated to fill space. In this
instance, the lattice sum will be summing over the atoms in the cell and
in nearby replica cells. Often, the cutoff distance is set such that we
consider only cells adjacent to the central simulation cell.

As we did before in a naive implementation of periodic boundary
conditions, we could sum over neighboring cells explicitly.

    \begin{tcolorbox}[breakable, size=fbox, boxrule=1pt, pad at break*=1mm,colback=cellbackground, colframe=cellborder]
\prompt{In}{incolor}{ }{\boxspacing}
\begin{Verbatim}[commandchars=\\\{\}]
\PY{c+c1}{\PYZsh{} NOTE: this naive implementation is very slow}
\PY{k}{def} \PY{n+nf}{lat\PYZus{}sum4}\PY{p}{(}\PY{n}{a}\PY{p}{,} \PY{n}{n}\PY{p}{,} \PY{n}{rc}\PY{p}{,} \PY{n}{c}\PY{p}{,} \PY{n}{s}\PY{p}{)}\PY{p}{:}
    \PY{l+s+sd}{\PYZdq{}\PYZdq{}\PYZdq{} Naive implementation of periodic boundary conditions}

\PY{l+s+sd}{        Input:}
\PY{l+s+sd}{          a (float): cell length}
\PY{l+s+sd}{          n (integer): number of atoms in simulation cell}
\PY{l+s+sd}{          rc (float): cutoff distance}
\PY{l+s+sd}{          c (integer): number of periodic neighbors to left and right}
\PY{l+s+sd}{          s (array): fractional coordinates of atom positions}
\PY{l+s+sd}{        Output:}
\PY{l+s+sd}{          ucell (float): total energy of simulation cell}
\PY{l+s+sd}{    \PYZdq{}\PYZdq{}\PYZdq{}}
    \PY{n}{start} \PY{o}{=} \PY{n}{time}\PY{o}{.}\PY{n}{process\PYZus{}time}\PY{p}{(}\PY{p}{)}
    \PY{n}{ucell} \PY{o}{=} \PY{l+m+mi}{0}
    \PY{k}{for} \PY{n}{i} \PY{o+ow}{in} \PY{n+nb}{range}\PY{p}{(}\PY{n}{n}\PY{p}{)}\PY{p}{:}
        \PY{k}{for} \PY{n}{j} \PY{o+ow}{in} \PY{n+nb}{range}\PY{p}{(}\PY{n}{n}\PY{p}{)}\PY{p}{:}
            \PY{k}{for} \PY{n}{k} \PY{o+ow}{in} \PY{n+nb}{range}\PY{p}{(}\PY{o}{\PYZhy{}}\PY{n}{c}\PY{p}{,} \PY{n}{c} \PY{o}{+} \PY{l+m+mi}{1}\PY{p}{)}\PY{p}{:}
                \PY{k}{for} \PY{n}{l} \PY{o+ow}{in} \PY{n+nb}{range}\PY{p}{(}\PY{o}{\PYZhy{}}\PY{n}{c}\PY{p}{,} \PY{n}{c} \PY{o}{+} \PY{l+m+mi}{1}\PY{p}{)}\PY{p}{:}
                    \PY{k}{for} \PY{n}{m} \PY{o+ow}{in} \PY{n+nb}{range}\PY{p}{(}\PY{o}{\PYZhy{}}\PY{n}{c}\PY{p}{,} \PY{n}{c} \PY{o}{+} \PY{l+m+mi}{1}\PY{p}{)}\PY{p}{:}
                        \PY{n}{xij} \PY{o}{=} \PY{n}{k} \PY{o}{+} \PY{n}{s}\PY{p}{[}\PY{n}{j}\PY{p}{,} \PY{l+m+mi}{0}\PY{p}{]} \PY{o}{\PYZhy{}} \PY{n}{s}\PY{p}{[}\PY{n}{i}\PY{p}{,} \PY{l+m+mi}{0}\PY{p}{]}
                        \PY{n}{yij} \PY{o}{=} \PY{n}{l} \PY{o}{+} \PY{n}{s}\PY{p}{[}\PY{n}{j}\PY{p}{,} \PY{l+m+mi}{1}\PY{p}{]} \PY{o}{\PYZhy{}} \PY{n}{s}\PY{p}{[}\PY{n}{i}\PY{p}{,} \PY{l+m+mi}{1}\PY{p}{]}
                        \PY{n}{zij} \PY{o}{=} \PY{n}{m} \PY{o}{+} \PY{n}{s}\PY{p}{[}\PY{n}{j}\PY{p}{,} \PY{l+m+mi}{2}\PY{p}{]} \PY{o}{\PYZhy{}} \PY{n}{s}\PY{p}{[}\PY{n}{i}\PY{p}{,} \PY{l+m+mi}{2}\PY{p}{]}
                        \PY{n}{dist} \PY{o}{=} \PY{n}{a} \PY{o}{*} \PY{n}{np}\PY{o}{.}\PY{n}{sqrt}\PY{p}{(}\PY{n}{xij}\PY{o}{*}\PY{o}{*}\PY{l+m+mi}{2} \PY{o}{+} \PY{n}{yij}\PY{o}{*}\PY{o}{*}\PY{l+m+mi}{2} \PY{o}{+} \PY{n}{zij}\PY{o}{*}\PY{o}{*}\PY{l+m+mi}{2}\PY{p}{)}
                        \PY{k}{if} \PY{l+m+mi}{0} \PY{o}{\PYZlt{}} \PY{n}{dist} \PY{o}{\PYZlt{}}\PY{o}{=} \PY{n}{rc}\PY{p}{:}
                            \PY{n}{phi} \PY{o}{=} \PY{l+m+mi}{4} \PY{o}{*} \PY{p}{(}\PY{l+m+mi}{1} \PY{o}{/} \PY{n}{dist}\PY{o}{*}\PY{o}{*}\PY{l+m+mi}{12} \PY{o}{\PYZhy{}} \PY{l+m+mi}{1} \PY{o}{/} \PY{n}{dist}\PY{o}{*}\PY{o}{*}\PY{l+m+mi}{6}\PY{p}{)}
                        \PY{k}{else}\PY{p}{:}
                            \PY{n}{phi} \PY{o}{=} \PY{l+m+mi}{0}
                        \PY{n}{ucell} \PY{o}{=} \PY{n}{ucell} \PY{o}{+} \PY{n}{phi}
    \PY{n}{ucell} \PY{o}{=} \PY{n}{ucell} \PY{o}{/} \PY{p}{(}\PY{l+m+mi}{2} \PY{o}{*} \PY{n}{n}\PY{p}{)}

    \PY{n}{end} \PY{o}{=} \PY{n}{time}\PY{o}{.}\PY{n}{process\PYZus{}time}\PY{p}{(}\PY{p}{)} \PY{o}{\PYZhy{}} \PY{n}{start}
    \PY{k}{return} \PY{n}{ucell}\PY{p}{,} \PY{n}{end}
 
\end{Verbatim}
\end{tcolorbox}

    Again, we can improve this code by noting that atoms on one side of a
cell are unlikely to interact (significantly) with atoms in a
neighboring cell on the opposite side. Thus, to avoid excess
computational cost, we want to avoid calculating distances to atoms that
are beyond the cutoff distance.

One strategy to accomplish this is to consider an imaginary box centered
around the atom for which one is summing the interactions. If the box is
set up so that its side has a length that is twice the cutoff distance
(i.e., on both sides), than all the atoms that can interact with the
atom at the center are within the box. One then only sums over the atoms
within the box. One still needs to check the cutoff criteria for cases
such as atoms at the box corners which will be outside of the cutoff
distance. More sophisticated methods, such as neighbor lists as
discussed in the text, may be used for large systems.

For the small systems considered here, we employ a method called the
\emph{minimum image convention}. We take the size of the imaginary box
to be the size of the simulation, so that the cutoff distance \(r_c\)
must be less than or equal to half the size of the simulation cell. This
way, each particle interacts only with the nearest image of the other
particles of the simulation cell. Using the minimum image convention
reduces the number of particles to search. Cutoff distances less than
half the box size can be used.

One reason the minimum image convention is an attractive strategy is its
ease of implementation. Assume we have a simulation cell with side of
length \(a\) and we want to find the neighbors of atom \(i\). We first
pick another atom \(j\) and compute \(r_{ij}\). We check to see if each
component of the \(r_{ij}\) lies inside or outside the cell with side
\(a\) centered on atom \(i\). If the component is outside that cell, we
add or substract a lattice vector to bring it back within the cell,
which will be accomplished with the \texttt{round(x)} function. We
create the image using the function

\texttt{x\ =\ x\ -\ round(x)}

Let us examine what \texttt{x\ =\ x\ -\ round(x)} does. If \(x > 1/2\),
\(x\) is replaced by \(x-1\). If \(x<-1/2\), \(x\) is replaced with
\(x+1\). Otherwise, \(x\) is unchanged. Thus, this function selects only
the nearest image of atom \(i\).

We can implement the minimum image convention by replacing the sum over
the lattice vectors.

    \begin{tcolorbox}[breakable, size=fbox, boxrule=1pt, pad at break*=1mm,colback=cellbackground, colframe=cellborder]
\prompt{In}{incolor}{31}{\boxspacing}
\begin{Verbatim}[commandchars=\\\{\}]
\PY{k}{def} \PY{n+nf}{lat\PYZus{}sum5}\PY{p}{(}\PY{n}{a}\PY{p}{,} \PY{n}{n}\PY{p}{,} \PY{n}{rc}\PY{p}{,} \PY{n}{s}\PY{p}{)}\PY{p}{:}
    \PY{l+s+sd}{\PYZdq{}\PYZdq{}\PYZdq{} Implementation of minimum image convention }

\PY{l+s+sd}{        Input:}
\PY{l+s+sd}{          a (float): cell length}
\PY{l+s+sd}{          n (integer): number of atoms in simulation cell}
\PY{l+s+sd}{          rc (float): cutoff distance}
\PY{l+s+sd}{          s (array): fractional coordinates of atom positions}
\PY{l+s+sd}{        Output:}
\PY{l+s+sd}{          ucell (float): total energy of simulation cell}
\PY{l+s+sd}{    \PYZdq{}\PYZdq{}\PYZdq{}}
    \PY{n}{start} \PY{o}{=} \PY{n}{time}\PY{o}{.}\PY{n}{process\PYZus{}time}\PY{p}{(}\PY{p}{)}
    \PY{n+nb}{print}\PY{p}{(}\PY{l+s+s2}{\PYZdq{}}\PY{l+s+s2}{starting lattice sum}\PY{l+s+s2}{\PYZdq{}}\PY{p}{)}

    \PY{n}{ucell} \PY{o}{=} \PY{l+m+mi}{0}
    \PY{k}{for} \PY{n}{i} \PY{o+ow}{in} \PY{n+nb}{range}\PY{p}{(}\PY{n}{n} \PY{o}{\PYZhy{}} \PY{l+m+mi}{1}\PY{p}{)}\PY{p}{:}
        \PY{k}{for} \PY{n}{j} \PY{o+ow}{in} \PY{n+nb}{range}\PY{p}{(}\PY{n}{i} \PY{o}{+} \PY{l+m+mi}{1}\PY{p}{,} \PY{n}{n}\PY{p}{)}\PY{p}{:}
            \PY{n}{xij} \PY{o}{=} \PY{n}{s}\PY{p}{[}\PY{n}{j}\PY{p}{,} \PY{l+m+mi}{0}\PY{p}{]} \PY{o}{\PYZhy{}} \PY{n}{s}\PY{p}{[}\PY{n}{i}\PY{p}{,} \PY{l+m+mi}{0}\PY{p}{]}
            \PY{n}{yij} \PY{o}{=} \PY{n}{s}\PY{p}{[}\PY{n}{j}\PY{p}{,} \PY{l+m+mi}{1}\PY{p}{]} \PY{o}{\PYZhy{}} \PY{n}{s}\PY{p}{[}\PY{n}{i}\PY{p}{,} \PY{l+m+mi}{1}\PY{p}{]}
            \PY{n}{zij} \PY{o}{=} \PY{n}{s}\PY{p}{[}\PY{n}{j}\PY{p}{,} \PY{l+m+mi}{2}\PY{p}{]} \PY{o}{\PYZhy{}} \PY{n}{s}\PY{p}{[}\PY{n}{i}\PY{p}{,} \PY{l+m+mi}{2}\PY{p}{]}
            \PY{n}{xij} \PY{o}{=} \PY{n}{xij} \PY{o}{\PYZhy{}} \PY{n+nb}{round}\PY{p}{(}\PY{n}{xij}\PY{p}{)}
            \PY{n}{yij} \PY{o}{=} \PY{n}{yij} \PY{o}{\PYZhy{}} \PY{n+nb}{round}\PY{p}{(}\PY{n}{yij}\PY{p}{)}
            \PY{n}{zij} \PY{o}{=} \PY{n}{zij} \PY{o}{\PYZhy{}} \PY{n+nb}{round}\PY{p}{(}\PY{n}{zij}\PY{p}{)}
            \PY{n}{dist} \PY{o}{=} \PY{n}{a} \PY{o}{*} \PY{n}{np}\PY{o}{.}\PY{n}{sqrt}\PY{p}{(}\PY{n}{xij}\PY{o}{*}\PY{o}{*}\PY{l+m+mi}{2} \PY{o}{+} \PY{n}{yij}\PY{o}{*}\PY{o}{*}\PY{l+m+mi}{2} \PY{o}{+} \PY{n}{zij}\PY{o}{*}\PY{o}{*}\PY{l+m+mi}{2}\PY{p}{)}
            \PY{k}{if} \PY{l+m+mi}{0} \PY{o}{\PYZlt{}} \PY{n}{dist} \PY{o}{\PYZlt{}}\PY{o}{=} \PY{n}{rc}\PY{p}{:}
                \PY{n}{phi} \PY{o}{=} \PY{l+m+mi}{4} \PY{o}{*} \PY{p}{(}\PY{l+m+mi}{1} \PY{o}{/} \PY{n}{dist}\PY{o}{*}\PY{o}{*}\PY{l+m+mi}{12} \PY{o}{\PYZhy{}} \PY{l+m+mi}{1} \PY{o}{/} \PY{n}{dist}\PY{o}{*}\PY{o}{*}\PY{l+m+mi}{6}\PY{p}{)}
            \PY{k}{else}\PY{p}{:}
                \PY{n}{phi} \PY{o}{=} \PY{l+m+mi}{0}
            \PY{n}{ucell} \PY{o}{=} \PY{n}{ucell} \PY{o}{+} \PY{n}{phi}
    \PY{n}{ucell} \PY{o}{=} \PY{n}{ucell} \PY{o}{/} \PY{n}{n}

    \PY{n}{end} \PY{o}{=} \PY{n}{time}\PY{o}{.}\PY{n}{process\PYZus{}time}\PY{p}{(}\PY{p}{)} \PY{o}{\PYZhy{}} \PY{n}{start}
    \PY{k}{return} \PY{n}{ucell}\PY{p}{,} \PY{n}{end}  
\end{Verbatim}
\end{tcolorbox}

    \hypertarget{lattice-statistics}{%
\subsection{Lattice statistics}\label{lattice-statistics}}

The lattice sums implemented above are the basis of a set of techniques
call \emph{lattice statistics}. In these simulations, the temperature is
0 K and the equilibrium atomic structure is determined by finding the
positions that minimize the potential energy of the system. For example,
we could take all the atoms in the positions of a perfect solid and vary
the lattice parameters to find the equilibrium structure. We can also
take a perfect solid and create a vacancy by removing one atom, and
subsequently find the structure and energy of that vacancy. We can also
vary the volume and plot the energy per unit volume in which that slope
of the resulting curve is the negative of the pressure at 0 K.

Such calculations typically involve lattice sums to evaluate the energy
and minimization routine to vary the atomic positions and/or lattice
parameters until the minimum energy configuration is reached. We shall
see a few of these cases in the exercises below.

    \hypertarget{exercises}{%
\section{Exercises}\label{exercises}}

    \hypertarget{perfect-cubic-solids}{%
\subsection{Perfect cubic solids}\label{perfect-cubic-solids}}

\begin{enumerate}
\def\labelenumi{\arabic{enumi}.}
\tightlist
\item
  Create a code to calculate the energy of a solid with interactions
  described by the Lennard-Jones potential. If appropriate, assume the
  minimum image convention. The input parameters should be the lattice
  constant \(a_0\), the number of repeated units \(n\), and the cutoff
  distance \(r_c\). The resulting lattice constant is \(a = n a_0\).
  Assume an FCC lattice and that \$\varepsilon = \sigma = 1 \$ in the
  Lennard-Jones potential.
\end{enumerate}

    \begin{enumerate}
\def\labelenumi{\arabic{enumi}.}
\setcounter{enumi}{1}
\tightlist
\item
  Vary the lattice constant and find the corresponding minimum energy
  and equilibrium lattice parameter.
\end{enumerate}

    \begin{enumerate}
\def\labelenumi{\arabic{enumi}.}
\setcounter{enumi}{2}
\tightlist
\item
  Vary the cutoff distance and find the associated error in the total
  energy calculations and lattice parameter for at least two values of
  the cutoff distance. Be sure to vary the cutoff such that you include
  different numbers of atomic shells in the sums. Create a plot of these
  values as a function of the number of neighboring shells. Discuss the
  differences you find with respect to the results with infinite sums.
\end{enumerate}

    \begin{enumerate}
\def\labelenumi{\arabic{enumi}.}
\setcounter{enumi}{3}
\tightlist
\item
  Repeat problems \#1-3 for the body-centered (BCC) structure. Note that
  the lattice positions will be different than those in FCC case.
  Determine the equilibrium energy for the BCC system and compare with
  that of the FCC system. Which structure is more stable (at 0 K)?
\end{enumerate}

    \begin{enumerate}
\def\labelenumi{\arabic{enumi}.}
\setcounter{enumi}{4}
\tightlist
\item
  For the FCC lattice, vary the lattice parameter and plot the energy vs
  volume. From this plot, determine the pressure and Gibbs free energy
  at 0 K. Plot the volume and Gibbs free energy as a function of the
  pressure.
\end{enumerate}


    % Add a bibliography block to the postdoc
    
    
    
\end{document}
